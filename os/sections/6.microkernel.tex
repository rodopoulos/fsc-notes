\chapter{\textit{Microkernel}}


Esta abordagem apresenta a desvantagem inata de possuir \textbf{baixo desempenho}. Tal característica provêm dê:
\begin{itemize}
  \item Aumento do número de tarefas que rodam no modo usuário;
  \item Aumento no número da troca de contextos;
  \item Mecanismo de IPC utilizado entre processos: \textit{send} e \textit{receive}.
\end{itemize}

Com o objetivo de mitigar este problema, a flexibilidade da arquitetura foi sacrificada, para que diversos serviços fossem re-integrados ao \textit{kernel}, evitando as trocas de contexto. Dessa forma, os mecanismos que permaneceram no \textit{kernel} foram: gerência de processos e \textit{threads}, comunicação de processos, proteção, gerência de memória dependente do \textit{hardware}, com a exclusão das operações de \textit{page-in} e \textit{page-out}.

\section{Tratamento de Entrada e Saída}
O espaço de endereçamento é a abstração óbvia para a incorporação de portas de dispositivos.
