\chapter{Entrada e Saída}

Para prover um abstração simples da máquina física, o sistema operacional deve conhecer bem cada \textit{hardware} sobre o qual ele opera, uma vez que cada um destes dispositivos tem um conjunto de comandos específicos. As unidades de entrada e saída são compostas por duas partes: uma parte mecânica, sendo o dispositivo propriamente dito, e uma parte eletrônica, que é a sua \textbf{controladora}, a qual irá interagir com o SO para realizar as operações de E/S.

\begin{figure}[H]
  \centering
  \includegraphics[width=.8\textwidth]{io-unity}
  \caption{Diversas unidades de entrada e saída do SO. Os dispositivos mecânicos estão ligados às suas respectivas controladoras.}
  \label{fig:io-unity}
\end{figure}

A \textbf{comunicação entre SO e controladora} se dá por registradores acessados pela CPU, onde o SO emite comandos escrevendo nestes e obtém informações sobre os dispositivos por meio da leitura dos registradores. Além disso, parte da comunicação também é feita por \textit{buffers}, onde o SO lê e escreve dados. Cada controladora irá possuir um conjunto de comandos específicos, definidos em seus manuais.

Os registradores de controle podem estar localizados nas controladores, sendo mapeados em portas de E/S. Entretanto, o sistema geralmente empregado é o de \textbf{E/S mapeado em memória}, onde os registradores são mapeados em posições fixas da memória, sendo associado um endereço a eles.

Em um processo genérico de E/S, temos que:
\begin{enumerate}
  \item A CPU envia um sinal ao barramento, contendo o endereço desejado, a operação (\textit{read}/\textit{write}) e espaço do registrador (controladora/memória);

  \item O dispositivo referênciado responde a requisição, emitindo uma chamada de sistema ao fim do serviço.
\end{enumerate}

Por fim, em geral, dividem-se os dispositivos de entrada e saída em dois tipos:
\begin{itemize}
  \item \textbf{Dispositivos de Bloco:} onde temos informações sendo armazenadas em unidades de tamanho fixo e sua recuperação é feita por acesso direto. \underline{Exemplo:} disco rídigo;

  \item \textbf{Dispositivos de Caractere:} a informação é aceita em unidadades de tamanho variável, Geralmente sendo armazenada temporariamente ou não sendo armazenada. \underline{Exemplo:} terminais, impressoras, \textit{mouse}, etc..
\end{itemize}





\section{\textit{Software de E/S}}
A maioria dos sistemas operacionais modernos utiliza o conceito de indepêndencia de dispositivos. Além disso, é importante haver \textbf{uniformidade da indentificação}, ou seja, o nome de um dispositivo lógico ou de um arquivo não deve depender do dispositivo físico ao qual está associado no momento.

\begin{definicao}{Independência de Dispositivos}
  Execução de um mesmo programa com dispositivos físicos diferentes.
\end{definicao}

\textit{Softwares} de E/S podem ser estruturados em: manipuladores de interrupção, \textit{drivers} de dispositivo, SO independente de dispositivo e aplicações de usuário.





\subsection{Manipuladores de Interrupção}
As operações de E/S podem ser tratadas de maneira síncrona ou assíncrona. A maioria dos SO tradicionais as trata de maneira síncrona, ou seja, o processo que solicitou uma operação de E/S fica bloqueado até que esta operação se complete.

Dessa forma, o SO tradicional pode tratar várias operações ao mesmo tempo, porém somente uma interrupção por processo pode estar sendo tratada. Quando ocorre uma interrupção, isto indica que a operação se completou e o SO coloca o processo, que estava bloqueado, na fila \textit{ready}.

\textbf{Nota:} as manipulações de interrupção devem ser feitas em modo protegido.




\subsection{\textit{Drivers} de Dispositivo}
O \textit{driver} de dispositivo é a parte do SO que executa código dependente de \textit{hardware} do dispositivo, onde cada um destes possui seu \textit{driver} específico. O \textit{driver} recebe solicitações abstratas e as traduz para comandos específicos.

No caso do disco rígido, o \textit{driver} de disco é o responsável por escrever os comandos da controladora do disco.




\subsection{\textit{Software Independente de Dispositivo}}
Existem operações de muito baixo nível que só podem ser executadas com comandos dependentes de dispositivo. A maioria dessas operações podem ser executada de maneira independente.

Uma operação que pode ser executada de maneira independente de dispositivo pode ser executada com comandos de baixo nível, por questões de performance. Assim, fica a cargo do projetista do SO que comandos serão independentes de dispositivo ou não. Temos os seguintes exemplos:
\begin{itemize}
  \item Fornecimento de uma interface uniforme ao usuário;

  \item Mapeamento do nome simbólico do dispositivo para o seu \textit{driver} específico;

  \item Proteção de acesso;

  \item Fornecimento de um tamanho de bloco independente do dispositivo.

  \item Tratamento de \textit{bufferização}, onde há a conversão de solicitações de tamanho arbitrário em solicitações de tamanho fixo;

  \item Alocação/liberação de blocos;

  \item Alocação/liberação de dispositivos de acesso exclusivo;

  \item Tratamento de erros não transientes.
\end{itemize}

\begin{figure}[h]
  \centering
  \includegraphics[width=0.35\textwidth]{es-layers}
  \caption{Camadas envolvidas em esquemas de entrada e saída. As camadas 2 à 4 são responsáveis pela gerência de E/S}
  \label{fig:es-layers}
\end{figure}
