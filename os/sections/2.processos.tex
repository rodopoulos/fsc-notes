\chapter{Fundamentos: Processos}

\textsc{Definição:} um programa em execução, composto por:
\begin{itemize}
  \item Código em execução;
  \item Pilha de execução e seu apontador;
  \item Contado de programa (\texttt{PC});
  \item Valores de registradores de máquina;
  \item E outras informações.
\end{itemize}

Cada processo possui um identificador único, conhecido como \textit{pid}. As informações do processo ficam armazenadas na \textbf{tabela de processos}, sendo o \textit{pid} de cada processo o seu indexador.

Durante a execução, o processo compartilha o processador com outros processos, sendo eles escalonados. A interação entre processos ocorre por mecanismos de comunicação próprios





\section{Modelo de Processos}
Classificamos os processos em relação ao custo de troca de contexto e manutenção:

\begin{itemize}
  \item \textbf{\textit{Heavyweight}:} são os processos tradicionais. Na sua entrada da tabela de processos temos tanto as informações de ambiente como as de execução;
  \item \textbf{\textit{Lightweight}:} são as threads. No processo thread, sua entrada na tabela de processos só contém as informações de ambiente.
\end{itemize}

Cada processo \textit{heaveweight} tem um único fluxo de controle, ou seja, em um dado momento ele só tem um valor de \texttt{PC}. Note que o processo \textit{heavyweight} pode contém vários processos \textit{threads}.
% TODO: checar a última frase

Todo sistema operacional deve possuir mecanismos que permitam a criação de processos. Geralmente, um processo somente é criado por outro processo, o que nos leva uma \textbf{hierarquia de árvore de processos}. Entretanto, note que a árvore de processos não é de fato implementada. A partir do campo de ID do processo pai (PPID), podemos virtualmente implementar essa árvore. Implementá-la de fato seria muito custoso.''







\section{Estado do Processo}

% TODO: colocar a imagem

Estados:
\begin{itemize}
  \item \textbf{Rodando:} processo em execução
  \item \textbf{Bloqueado:} processo parado, aguardando alguma coisa;
  \item \textbf{Pronto:} processo parado, pronto para ser executado, aguardando sua vez.
\end{itemize}


\section{Tabela de Processos}
Notas:
\begin{itemize}
  \item campos da tabela de processos variam de sistema operacional
  \item Dados referentes a memória:
  \begin{itemize}
    \item o processo tem no mínimo 3 áreas: código, dados e pilha;
    \item segmento de texto = código
    \item o segmento de texto normalmente é read-only, mas em alguns sistemas podemos por read-write
    \begin{itemize}
      \item essa estratégia é interessante em sistemas com área de memória pequena.
    \end{itemize}
  \end{itemize}
\end{itemize}

\subsection{Troca de Contexto}
Notas:
\begin{itemize}
  \item Valores de registradores = contexto.
\end{itemize}


\section{Escalonamento de Processos}


\subsection{Critérios de Escalonamento}
Critérios:
\begin{itemize}
  \item \textbf{\textit{Fairness}:} garantir que todos os processos do sistema terão chances de uso do processador, ou seja, \textbf{não há \textit{starvation}}. Garantir que os processos tenham chances \textit{iguais}, é algo muito forte e não usamos essa definição;

  \item \textbf{Definição:} se há demanda, a CPU deve estar ocupada.

  \item \textbf{Outros:} minimizar o tempo de resposta, minimizar o \textit{waiting time}, maximizar o \textit{throughtput}, etc..
  \begin{itemize}
    \item minimizar o tempo de resposta: % TODO: ver o que é isso
    \item waiting time: tempo entre os estados de \textit{Ready} e \textit{Running}
    \item Maximizar o throughtput: aumentar o número de processos concluídos em um tempo
  \end{itemize}
\end{itemize}


\subsection{Classificação de Escalonadores}
VER SLIDES

Notas:
\begin{itemize}
  \item escalonadores não-preemptivos: violam todos os critérios de um bom escalonador
\end{itemize}


\subsection{Algoritmo de Escalonamento}

Notas:
\begin{itemize}
  \item É o algoritmo de escalonamento que determina a mudança de prioridade dos processos

  \item: seria interessante ter um FCFS preemptivo: sim! ele se chama Round-Robin.

  \item Escalonamento com prioridade: um processo pode monopolizar a CPU % TODO: ver o porquê.

  \item SJF: não é muito justo, pois um processo pode entrar em starvation caso os processos que entrem na fila sejam sempre menores que ele.
\end{itemize}



\section{\textit{Threads}}


\section{\textit{Deadlock}}
\begin{definicao}{\textit{Deadlock}}
  Um conjunto de processos está em \textit{deadlock} se cada processo pertencente ao conjunto estiver esperando por um evento que somente um outro processo pertencente ao mesmo conjunto pode fazer ocorrer. Ou seja, \textbf{é uma espera eterna por recursos}.
\end{definicao}

\begin{definicao}{\textit{Livelock}}
  ocorre quando um conjunto de processos estão em parte de sua execução que é um loop eterno do qual nunca saírão, geralmente fazendo \textit{pooling}
\end{definicao}

\textbf{Nota:} Do ponto de vista teórico, livelock e deadlock tem significados equivalentes.

\begin{definicao}{\textit{Starvation}}
  Ocorre quando um conjunto de processos é indefinidamente preterido porque sua prioridade é menor que a prioridade outros conjuntos de processos. Ou seja, é \textbf{uma espera por um tempo indefinido por recursos}.
\end{definicao}










\section{Comunicação entre Procesos}
Processos interagem entre si, trocando informações. Dito isso, temos dois tipos de trocas de informações, ou cooperações: por compartilhamento de variáveis ou por troca de mensagens.

\subsection{Compartilhamento de Variáveis}
Um processo escreve um valor em uma posição de memória e outro processo lê este mesmo valor. Como o sistema operacional determina, através da política de escalonamento, o processo que vai rodar e por quanto tempo, não sabemos a priori em
que ordem dois processos ativos irão se executar. Daí, podemos ter inconsistências. % TODO: pegar imagem do slide

\begin{definicao}{Condição de Corrida}
  Quando dois ou mais processos acessam concorrentemente as mesmas posições de memória e o valor final contido nestas posições \textbf{dependem da ordem} na qual os processos foram executados, então temos uma condição de corrida.
\end{definicao}

Condição de corrida pode não ser um erro ou algo indesejável. Alguns programadores podem tirar proveito disso. Além disso, podemos definir que os processos escrevam na mesma posição, porém forçando ser o mesmo valor. Neste último caso, \textit{não temos condição de corrida}.


\subsection{Exclusão Mútua}
Para evitar as condições de corrida, devemos estabelecer que os processos sejam executados segunda certa ordem, de maneira exclusiva.

\subsubsection{Espera Ocupada}

\subsubsection{Bloqueio de Processos}
